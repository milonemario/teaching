\documentclass{article}
\usepackage[T1]{fontenc}
\usepackage[utf8]{inputenc}
\usepackage[margin=1.5in]{geometry}

\usepackage{comment, makecell, hyperref, graphicx, fancyhdr, xcolor}

\hypersetup{
    colorlinks=true,
    linkcolor=blue,
    urlcolor=blue,
}

\fancyhf{}
\renewcommand{\headrulewidth}{0pt}
\fancyhead[L]{
	\vspace*{-1in}
	\hspace*{-1in}
  \includegraphics[height=.8in]{rady-logo.png}
}
\lfoot{
	\vspace*{.5in}
	\hspace*{-1in}
	%\fontfamily{lmss}\selectfont
	\color{gray}
	\copyright \hspace{1pt} Mario Milone, 2021. Do not copy or distribute without permission.
}

\pagestyle{fancy}


\title{Syllabus -- MGTP 495 \\ Strategic Cost Management \& New Technologies}
\author{Mario Milone \\ \\
	University of California, San Diego \\
	Rady School of Management}
\date{}

\definecolor{colorpart}{rgb}{0, 0, 0.6}
\definecolor{colorsession}{rgb}{0.6, 0, 0}
\newcommand{\session}[6]{
\subsubsection*{\color{colorsession}Session #1 -- #2}
\begin{tabular}{p{.8in}p{1.5in}p{1.5in}l}
\bf{Date}	& #3 	& \bf{Readings} & #4 \\
\bf{Case Study} & \emph{#5} & \bf{Topic case study} & #6 \\
\end{tabular} \\
%\vspace{5pt}
}

\begin{document}
\maketitle
\thispagestyle{fancy}

\emph{Note: The syllabus may be subject to minor changes regarding its
outline.}

\section*{Course Information}
\begin{tabular}{p{1.5in}l}
\bf{Course Number}	&	MGTP 495 \\
\bf{Course Title}	&	Strategic Cost Management \& New Technologies \\
\bf{Quarter}		&	Spring 2021 \\
\bf{Class meetings}	&	Thursdays, 9:00am to 11:50am \\
\bf{Classroom}		&	Zoom \\
\bf{Zoom ID}		& 	974 2293 7250 \\
\bf{Zoom Password}	&	320691 \\
\bf{Zoom Link}		& 	\url{https://ucsd.zoom.us/j/97422937250?pwd=UnVNVHovMkJVQUlNWEM1am1YczVGQT09} \\
\end{tabular}

\section*{Contact Information}
\begin{tabular}{p{1.5in}l}
\bf{Professor}			&	Mario Milone \\
\bf{Office}					&	Remote (Zoom) \\
\bf{Email Address}	&	mmilone@ucsd.edu \\
\bf{Office Hours}		&	By appointment \\
\end{tabular}

\section*{Textbooks and Readings}
\begin{tabular}[t]{p{1.5in}l}
\bf{Textbook}		&\makecell[l]{\emph{Managerial Accounting: Making
			Decisions and} \\ \emph{Motivating Performance} \\
			by S. Datar and M. Rajan \\
			Pearson, 2014, ISBN 978-0-13-702487-2} \\
\bf{Case Studies}	&	See the outline below. \\
\end{tabular}

\newpage

\section*{Course Description}
This course focuses on the use of internal accounting system for budgeting,
planning and decision-making.
The first part of the course will provide the vocabulary and tools needed to
the understanding of costs, planning, budgeting and pricing decisions.
The second part of the course will focus on how to extract relevant information
for performance evaluation and control.

The vocabulary and skills developed in this course are essential for managers,
consultants, accountants, as they provide a better understanding of the
organization's internal operations. Being able to extract and analyze
relevant accounting information for decision-making purposes is extremely
valuable and applicable in many settings, including finance, accounting,
marketing, operations and strategy.

The concepts will be both introduced in class as well as practiced and
discussed through an extensive set of case studies.

\section*{Course Objectives}

At the close of this course, you will be able to:
\begin{itemize}
	\item Understand the different cost concepts.
	\item Perform simple and complex cost analyses.
	\item Understand pricing decisions.
	\item Perform profitability analyses.
	\item Understand performance evaluations.
	\item Understand transfer pricing and its implications.
\end{itemize}

\section*{Course Preparation and Requirements}
The course is heavily based on the use of case studies that students need to
prepare every week as assignments.
The concepts used in each case study is presented in class before the assignments
(one or two weeks before).
The outline below precises which concepts are used for each case study (``Topic case study'').
Students are required to read the book chapters mentioned in the outline (see below)
{\bf before} the class. This requirement also holds for the first session of the course.
Therefore, {\bf students are required to read chapters 1, 2, 4 \& 5 before the course
starts}.


\section*{Course Evaluation}

Evaluation in this course will be based on group case studies and a final individual case study.
The final grade will be determined as follows:

\vspace{.2in}
\begin{tabular}{p{2in}l}
1 Group Case Studies			&	50\% \\
2 Final Individual Case Study		&	50\% \\
3 Bonus points				&	2\%
\end{tabular}
\vspace{.2in}

A {\bf bonus} of 2\% will be added for students that fill in the professor evaluation form.

\subsubsection*{1 Group Case Assignments}
Students are required to prepare cases and hand-in written assignements at
every class (except the first one).
For each case, the assignment with questions will be posted on Canvas the week before.
These questions are designed to help students understand the case and ensure that
they are ready to discuss the case during class.
Assignments can be done by groups of maximum 3 students and should be
prepared as a \emph{single PDF document}. The document should be kept as brief
as possible. When applicable, an appendix containing quantitative analyses can be included.
Assignments should be uploaded to Canvas \emph{before} the beginning of the
class. They will be graded from 0 to 100. While approximately 9 assignements
will be required and graded, only the 6 highest grades will determine
the overall score for the group case studies.

\subsubsection*{2 Final Case Study}
The final exam will be in the form of a case study that students will have to
individually analyse in its entirety.
More information will be provided as we approach the end of quarter.

\section*{Dynamics of the syllabus}
In order to promote the best learnings experience, this syllabus is subject to
changes as the course progresses. Any change will be made in the interest of
the students.

\section*{Course Outline}

\subsection*{\color{colorpart}Part I -- Cost Accounting and Decision Making}

\session{1}{Introduction and Concepts \& Product Costing}{April 1}{Chapters 1, 2, 4 \& 5}{No Case Study this week}{NA}

There is no assignment for this class.
\begin{itemize}
	\item	Part I: Introduction and Concepts (chapters 1 \& 2)
	\item	Part II: Product Costing (chapters 3 \& 4)
\end{itemize}

\session{2}{CVP Analysis}{April 8}
{Chapters 3 \& 8}{Precision Worldwide}{Relevant Costs}

\session{3}{Activity Based Costing}{April 15}
{Chapter 6}{Bridgeton Industries}{Product Costing}

\session{4}{Pricing Decisions}{April 22}
{Chapter 7}{Seligram}{CVP Analysis}

\subsection*{\color{colorpart}Part II -- Performance Evaluation and Control}

\session{5}{Profitability Analysis}{April 29}
{Chapter 14}{Wilkerson}{Activity Based Costing}

\session{6}{Variance Analysis}{May 6}
{Chapters 13}{Owens \& Minor (A)}{Pricing Decisions}

\session{7}{Control and Performance Evaluation}{May 13}
{Chapters 16}{Store 24}{Profitability Analysis}

\session{8}{Case Study: Compagnie du Froid}{May 20}
{NA}{Compagnie du Froid}{Variance Analysis}

\session{9}{Transfer Pricing I}{May 27}
{Chapter 15}{Vyaderm}{Control and Performance Evaluation}

\session{10}{Transfer Pricing II \& Conclusion}{June 3}
{Chapter 15}{Birch Paper Company}{Transfer Pricing}

\newpage
\section*{Academic Integrity}

Integrity of scholarship is essential for an academic community. As members of the Rady School, we pledge ourselves to uphold the highest ethical standards. The University expects that both faculty and students will honor this principle and in so doing protect the validity of University intellectual work. For students, this means that all academic work will be done by the individual to whom it is assigned, without unauthorized aid of any kind.

The complete UCSD Policy on Integrity of Scholarship can be viewed at:
\url{http://senate.ucsd.edu/Operating-Procedures/Senate-Manual/Appendices/2}

\section*{Students with disabilities}

A student who has a disability or special need and requires an accommodation in order to have equal access to the classroom must register with the Office for Students with Disabilities (OSD). The OSD will determine what accommodations may be made and provide the necessary documentation to present to the faculty member.

The student must present the OSD letter of certification and OSD accommodation recommendation to the appropriate faculty member in order to initiate the request for accommodation in classes, examinations, or other academic program activities. No accommodations can be implemented retroactively.

Please visit the \href{http://disabilities.ucsd.edu/about/index.html}{OSD website} for further information or contact the Office for Students with Disabilities at \href{tel:(858) 534-4382}{(858) 534-4382} or \href{mailto:osd@ucsd.edu}{osd@ucsd.edu}.


\end{document}
